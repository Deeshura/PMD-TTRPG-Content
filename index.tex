% Options for packages loaded elsewhere
% Options for packages loaded elsewhere
\PassOptionsToPackage{unicode}{hyperref}
\PassOptionsToPackage{hyphens}{url}
%
\documentclass[
  11pt,
  letterpaper,
]{scrbook}
\usepackage{xcolor}
\usepackage[paperwidth=148mm,paperheight=210mm,left=15mm,right=15mm,top=20mm,bottom=25mm]{geometry}
\usepackage{amsmath,amssymb}
\setcounter{secnumdepth}{5}
\usepackage{iftex}
\ifPDFTeX
  \usepackage[T1]{fontenc}
  \usepackage[utf8]{inputenc}
  \usepackage{textcomp} % provide euro and other symbols
\else % if luatex or xetex
  \usepackage{unicode-math} % this also loads fontspec
  \defaultfontfeatures{Scale=MatchLowercase}
  \defaultfontfeatures[\rmfamily]{Ligatures=TeX,Scale=1}
\fi
\usepackage{lmodern}
\ifPDFTeX\else
  % xetex/luatex font selection
  \setmainfont[]{EB Garamond}
\fi
% Use upquote if available, for straight quotes in verbatim environments
\IfFileExists{upquote.sty}{\usepackage{upquote}}{}
\IfFileExists{microtype.sty}{% use microtype if available
  \usepackage[]{microtype}
  \UseMicrotypeSet[protrusion]{basicmath} % disable protrusion for tt fonts
}{}
\makeatletter
\@ifundefined{KOMAClassName}{% if non-KOMA class
  \IfFileExists{parskip.sty}{%
    \usepackage{parskip}
  }{% else
    \setlength{\parindent}{0pt}
    \setlength{\parskip}{6pt plus 2pt minus 1pt}}
}{% if KOMA class
  \KOMAoptions{parskip=half}}
\makeatother
% Make \paragraph and \subparagraph free-standing
\makeatletter
\ifx\paragraph\undefined\else
  \let\oldparagraph\paragraph
  \renewcommand{\paragraph}{
    \@ifstar
      \xxxParagraphStar
      \xxxParagraphNoStar
  }
  \newcommand{\xxxParagraphStar}[1]{\oldparagraph*{#1}\mbox{}}
  \newcommand{\xxxParagraphNoStar}[1]{\oldparagraph{#1}\mbox{}}
\fi
\ifx\subparagraph\undefined\else
  \let\oldsubparagraph\subparagraph
  \renewcommand{\subparagraph}{
    \@ifstar
      \xxxSubParagraphStar
      \xxxSubParagraphNoStar
  }
  \newcommand{\xxxSubParagraphStar}[1]{\oldsubparagraph*{#1}\mbox{}}
  \newcommand{\xxxSubParagraphNoStar}[1]{\oldsubparagraph{#1}\mbox{}}
\fi
\makeatother

\usepackage{color}
\usepackage{fancyvrb}
\newcommand{\VerbBar}{|}
\newcommand{\VERB}{\Verb[commandchars=\\\{\}]}
\DefineVerbatimEnvironment{Highlighting}{Verbatim}{commandchars=\\\{\}}
% Add ',fontsize=\small' for more characters per line
\usepackage{framed}
\definecolor{shadecolor}{RGB}{241,243,245}
\newenvironment{Shaded}{\begin{snugshade}}{\end{snugshade}}
\newcommand{\AlertTok}[1]{\textcolor[rgb]{0.68,0.00,0.00}{#1}}
\newcommand{\AnnotationTok}[1]{\textcolor[rgb]{0.37,0.37,0.37}{#1}}
\newcommand{\AttributeTok}[1]{\textcolor[rgb]{0.40,0.45,0.13}{#1}}
\newcommand{\BaseNTok}[1]{\textcolor[rgb]{0.68,0.00,0.00}{#1}}
\newcommand{\BuiltInTok}[1]{\textcolor[rgb]{0.00,0.23,0.31}{#1}}
\newcommand{\CharTok}[1]{\textcolor[rgb]{0.13,0.47,0.30}{#1}}
\newcommand{\CommentTok}[1]{\textcolor[rgb]{0.37,0.37,0.37}{#1}}
\newcommand{\CommentVarTok}[1]{\textcolor[rgb]{0.37,0.37,0.37}{\textit{#1}}}
\newcommand{\ConstantTok}[1]{\textcolor[rgb]{0.56,0.35,0.01}{#1}}
\newcommand{\ControlFlowTok}[1]{\textcolor[rgb]{0.00,0.23,0.31}{\textbf{#1}}}
\newcommand{\DataTypeTok}[1]{\textcolor[rgb]{0.68,0.00,0.00}{#1}}
\newcommand{\DecValTok}[1]{\textcolor[rgb]{0.68,0.00,0.00}{#1}}
\newcommand{\DocumentationTok}[1]{\textcolor[rgb]{0.37,0.37,0.37}{\textit{#1}}}
\newcommand{\ErrorTok}[1]{\textcolor[rgb]{0.68,0.00,0.00}{#1}}
\newcommand{\ExtensionTok}[1]{\textcolor[rgb]{0.00,0.23,0.31}{#1}}
\newcommand{\FloatTok}[1]{\textcolor[rgb]{0.68,0.00,0.00}{#1}}
\newcommand{\FunctionTok}[1]{\textcolor[rgb]{0.28,0.35,0.67}{#1}}
\newcommand{\ImportTok}[1]{\textcolor[rgb]{0.00,0.46,0.62}{#1}}
\newcommand{\InformationTok}[1]{\textcolor[rgb]{0.37,0.37,0.37}{#1}}
\newcommand{\KeywordTok}[1]{\textcolor[rgb]{0.00,0.23,0.31}{\textbf{#1}}}
\newcommand{\NormalTok}[1]{\textcolor[rgb]{0.00,0.23,0.31}{#1}}
\newcommand{\OperatorTok}[1]{\textcolor[rgb]{0.37,0.37,0.37}{#1}}
\newcommand{\OtherTok}[1]{\textcolor[rgb]{0.00,0.23,0.31}{#1}}
\newcommand{\PreprocessorTok}[1]{\textcolor[rgb]{0.68,0.00,0.00}{#1}}
\newcommand{\RegionMarkerTok}[1]{\textcolor[rgb]{0.00,0.23,0.31}{#1}}
\newcommand{\SpecialCharTok}[1]{\textcolor[rgb]{0.37,0.37,0.37}{#1}}
\newcommand{\SpecialStringTok}[1]{\textcolor[rgb]{0.13,0.47,0.30}{#1}}
\newcommand{\StringTok}[1]{\textcolor[rgb]{0.13,0.47,0.30}{#1}}
\newcommand{\VariableTok}[1]{\textcolor[rgb]{0.07,0.07,0.07}{#1}}
\newcommand{\VerbatimStringTok}[1]{\textcolor[rgb]{0.13,0.47,0.30}{#1}}
\newcommand{\WarningTok}[1]{\textcolor[rgb]{0.37,0.37,0.37}{\textit{#1}}}

\usepackage{longtable,booktabs,array}
\usepackage{calc} % for calculating minipage widths
% Correct order of tables after \paragraph or \subparagraph
\usepackage{etoolbox}
\makeatletter
\patchcmd\longtable{\par}{\if@noskipsec\mbox{}\fi\par}{}{}
\makeatother
% Allow footnotes in longtable head/foot
\IfFileExists{footnotehyper.sty}{\usepackage{footnotehyper}}{\usepackage{footnote}}
\makesavenoteenv{longtable}
\usepackage{graphicx}
\makeatletter
\newsavebox\pandoc@box
\newcommand*\pandocbounded[1]{% scales image to fit in text height/width
  \sbox\pandoc@box{#1}%
  \Gscale@div\@tempa{\textheight}{\dimexpr\ht\pandoc@box+\dp\pandoc@box\relax}%
  \Gscale@div\@tempb{\linewidth}{\wd\pandoc@box}%
  \ifdim\@tempb\p@<\@tempa\p@\let\@tempa\@tempb\fi% select the smaller of both
  \ifdim\@tempa\p@<\p@\scalebox{\@tempa}{\usebox\pandoc@box}%
  \else\usebox{\pandoc@box}%
  \fi%
}
% Set default figure placement to htbp
\def\fps@figure{htbp}
\makeatother





\setlength{\emergencystretch}{3em} % prevent overfull lines

\providecommand{\tightlist}{%
  \setlength{\itemsep}{0pt}\setlength{\parskip}{0pt}}



 


\makeatletter
\@ifpackageloaded{tcolorbox}{}{\usepackage[skins,breakable]{tcolorbox}}
\@ifpackageloaded{fontawesome5}{}{\usepackage{fontawesome5}}
\definecolor{quarto-callout-color}{HTML}{909090}
\definecolor{quarto-callout-note-color}{HTML}{0758E5}
\definecolor{quarto-callout-important-color}{HTML}{CC1914}
\definecolor{quarto-callout-warning-color}{HTML}{EB9113}
\definecolor{quarto-callout-tip-color}{HTML}{00A047}
\definecolor{quarto-callout-caution-color}{HTML}{FC5300}
\definecolor{quarto-callout-color-frame}{HTML}{acacac}
\definecolor{quarto-callout-note-color-frame}{HTML}{4582ec}
\definecolor{quarto-callout-important-color-frame}{HTML}{d9534f}
\definecolor{quarto-callout-warning-color-frame}{HTML}{f0ad4e}
\definecolor{quarto-callout-tip-color-frame}{HTML}{02b875}
\definecolor{quarto-callout-caution-color-frame}{HTML}{fd7e14}
\makeatother
\makeatletter
\@ifpackageloaded{bookmark}{}{\usepackage{bookmark}}
\makeatother
\makeatletter
\@ifpackageloaded{caption}{}{\usepackage{caption}}
\AtBeginDocument{%
\ifdefined\contentsname
  \renewcommand*\contentsname{Table of contents}
\else
  \newcommand\contentsname{Table of contents}
\fi
\ifdefined\listfigurename
  \renewcommand*\listfigurename{List of Figures}
\else
  \newcommand\listfigurename{List of Figures}
\fi
\ifdefined\listtablename
  \renewcommand*\listtablename{List of Tables}
\else
  \newcommand\listtablename{List of Tables}
\fi
\ifdefined\figurename
  \renewcommand*\figurename{Figure}
\else
  \newcommand\figurename{Figure}
\fi
\ifdefined\tablename
  \renewcommand*\tablename{Table}
\else
  \newcommand\tablename{Table}
\fi
}
\@ifpackageloaded{float}{}{\usepackage{float}}
\floatstyle{ruled}
\@ifundefined{c@chapter}{\newfloat{codelisting}{h}{lop}}{\newfloat{codelisting}{h}{lop}[chapter]}
\floatname{codelisting}{Listing}
\newcommand*\listoflistings{\listof{codelisting}{List of Listings}}
\makeatother
\makeatletter
\makeatother
\makeatletter
\@ifpackageloaded{caption}{}{\usepackage{caption}}
\@ifpackageloaded{subcaption}{}{\usepackage{subcaption}}
\makeatother
\usepackage{bookmark}
\IfFileExists{xurl.sty}{\usepackage{xurl}}{} % add URL line breaks if available
\urlstyle{same}
\hypersetup{
  pdftitle={World of PKMN},
  pdfauthor={Deeshura},
  hidelinks,
  pdfcreator={LaTeX via pandoc}}


\title{World of PKMN}
\author{Deeshura}
\date{2025-05-16}
\begin{document}
\frontmatter
\maketitle

\renewcommand*\contentsname{Table of contents}
{
\setcounter{tocdepth}{2}
\tableofcontents
}

\mainmatter
\bookmarksetup{startatroot}

\chapter*{Welcome}\label{welcome}
\addcontentsline{toc}{chapter}{Welcome}

\markboth{Welcome}{Welcome}

Welcome! This book is the portal that leads to the World of Pokemon!

\ldots To be a \emph{little} more specific, this book is a guide to
playing ``World of PKMN'', a tabletop roleplaying game where a group of
players work together to explore mysterious dungeons and overcome
challenges in pursuit of fame and fortune, under the guidance of a Game
Master.

Your journeys will take place in the World of Pokemon, a land rife with
ancient ruins, deep caverns, sprawling forests. A world of endless
adventure awaits- be strong! Stay smart! And be victorious!

\begin{tcolorbox}[enhanced jigsaw, toptitle=1mm, leftrule=.75mm, bottomrule=.15mm, colbacktitle=quarto-callout-important-color!10!white, opacitybacktitle=0.6, toprule=.15mm, colframe=quarto-callout-important-color-frame, breakable, rightrule=.15mm, titlerule=0mm, left=2mm, coltitle=black, title=\textcolor{quarto-callout-important-color}{\faExclamation}\hspace{0.5em}{Disclaimer}, bottomtitle=1mm, opacityback=0, arc=.35mm, colback=white]

This is a fan-made tabletop roleplaying game intended for free use.
Nintendo and Game Freak own all rights to Pokémon and related media. By
reading this book, you agree to not copy any Pokemon-related material
for profit or resale. Neither you nor I want to provoke the wrath of the
Nintendo ninjas.

\end{tcolorbox}

\bookmarksetup{startatroot}

\chapter*{Preface}\label{preface}
\addcontentsline{toc}{chapter}{Preface}

\markboth{Preface}{Preface}

There have been many attempts to adapt Pokemon and its dungeon crawling
spin-off Pokemon Mystery Dungeon (henceforth, PMD) to a TTRPG format.
The reason why is obvious; PMD is beloved by its fans for its shift in
focus compared to the main series\footnote{That, or because it enables
  people's furry fantasies. Up to you.}, and a TTRPG adaption would
allow fans to create as many new adventures in that world as they
desire.

For myself, personally, many of these attempts to adapt PMD have fallen
short of being the perfect game. Of course, everyone will have a
different vision of what the ``perfect'' adaption of PMD would be; what
some tabletop players may find enjoyable, others may not particularly
gel with.

Nevertheless, unsatisfied with what exists, I set out to create what I
envisioned to be the perfect adaption of PMD into a tabletop format.

\newpage

\section*{What is PMD?}\label{what-is-pmd}
\addcontentsline{toc}{section}{What is PMD?}

\markright{What is PMD?}

Of course, the question at the heart of trying to create the perfect PMD
adaption is ``what even \emph{is} PMD about anyway?''.

Pokemon Mystery Dungeon, as its name implies, is a crossover of two
different franchises; Pokemon, a creature-collecting RPG that is easy to
learn, but with a huge variety of options at the players disposal, and
the ``Mystery Dungeon'' series of games, which are very focused and
straightforward rogue-like dungeon crawlers. Yet, this crossover
downplays the core qualities of both its halves, with less focus on
creature collecting and challenging dungeon crawls, and instead ups the
focus on narrative, featuring stories with a strong emotional core that
touched the hearts of fans around the world.

In some ways, this is utterly incoherent, yet nevertheless this strong
narrative focus is what has allowed PMD to carve out a dedicated and
loyal fan-base separate from that of the main Pokemon series. A fully
narrative-focused game with Pokemon-characters as set dressing could
work fine; I've seen PMD-themed games of Blades in the Dark been run
before, for example.

That being said, I feel like an approach that is \emph{fully} narrative
focused does a disservice to its predecessors and the history of PMD and
its creation. PMD is not quite at the stage of being a visual novel; the
dungeon crawls, love them or hate them, are a part of every game that no
player can really ignore.

Thus, this book sets out to be the following; a rules-light, low-stakes
dungeon-crawler that tastefully implements Pokemon mechanics in a way
that supports storytelling and narrative-focused gameplay.

\newpage

\section*{\texorpdfstring{What this game
\emph{is}}{What this game is}}\label{what-this-game-is}
\addcontentsline{toc}{section}{What this game \emph{is}}

\markright{What this game \emph{is}}

A fantasy-esque RPG where a group of Players work alongside a GM to tell
a story of adventure, in a world where \emph{generally} all player and
non-player characters (PCs and NPCs) are talking Pokemon\footnote{We are
  not responsible if your GM decides to swerve the plot towards a giant
  evil orb with a glowing red core.}.

The rules should, ideally, be relatively ``light'', though creating a
fully rules-light system whilst sticking to Pokemon's conventions
alongside the expectations of the dungeon-crawling genre is challenging.
Nevertheless, the ``crunch'' in this system is designed to help it reach
its goal of being a strong PMD adaption, and should ideally not get in
the way of the players or GM.

\section*{\texorpdfstring{What this game
\emph{isn't}}{What this game isn't}}\label{what-this-game-isnt}
\addcontentsline{toc}{section}{What this game \emph{isn't}}

\markright{What this game \emph{isn't}}

Pokemon, are, at their core, their own individual characters. Two
different Pikachu could have completely separate motivations, goals,
personalities, and movesets. With this in mind, and to be frank, for the
sake of levity, there are no species-specific rules in this game, nor an
expansive included Pokedex. The rules instead focus on supporting GM's
and Players to build their own Pokemon quickly and easily.

Whilst this book borrows quite heavily from the ``Old School
Renaissance'' (henceforth, OSR) scene in terms of dungeon-crawling
design, in order to better adapt PMD's tone, there is much less focus on
lethal consequences and crushing failures here compared to other OSR
games. In fact, there are no rules for dying at all- defeat can be
consequential, but it will rarely ever spell a character's demise.

This game is not going to be a masterpiece of graphic design. In fact,
it's in some ways little more than a glorified Obsidian notebook, being
markdown that's then converted into a web-book and an accompany PDF via
Quarto. I do not have the time nor the money to put together any kind of
hardcover product-nor do I want The Pokemon Company sending lawyers to
my doorstep!

\bookmarksetup{startatroot}

\chapter{Introduction}\label{introduction}

A tabletop roleplaying game (henceforth, ``TTRPG'') is an interactive
storytelling game where one player, the ``Game Master'' (``GM'') acts as
the narrator and referee, being in charge of describing the scene and
presenting challenges to the other players, who take the role of the
``Player Characters'' (PCs). The players must overcome thrilling battles
and labyrinthine dungeons filled with devious traps- all in search of
fame and fortune.

The game will typically be playing in a group of around 5 players, with
one of said players serving as the group's GM. The GM is responsible for
preparing adventures for the party to undertake, and guiding how the
world responds to the actions of the player characters, taking control
of rivals and allies alike. Each player has a role in contributing to
the story, and dice rolls are used to add elements of risk to the taken
actions.

\section{Rule Number One}\label{rule-number-one}

The very first rule of this TTRPG is to have fun\footnote{Some fairy
  types may claim rule number one is actually ``Don't think about it!''.
  Let this be a reminder to never take advice from the fair folk.}. Your
group has the right to modify the rest of the rules for your personal
use in order to accomplish this. Don't like how something is written
here? Feel free to do something different, or ignore it entirely. Your
journey is for \emph{you} to decide, after all.

\section{Rule Number Two}\label{rule-number-two}

The next most important rule is that rulings from the GM take precedent
over the rules in this book. It's impossible for this book to offer
rules and guidance on literally every situation that can occur at the
table- so when a situation that isn't explained in this book crops up,
the GM should rule based on what makes narrative (or at the very least,
common) sense. See rule number one; the primary goal of the game is not
to deconstruct the rules like it's the bar exam, but to tell a
fulfilling story and have fun; and it's the GM's responsibility to make
rulings to support that.

\section{The Flow of the Game}\label{the-flow-of-the-game}

TTRPG's are typically played in sessions, during which players gather
for a few hours and play the game. Sessions can take place in person
around a table with what they need to play physically with them, or
alternatively, can be played online using digital toolsets and platforms
that emulate the in-person experience. Some groups even play tabletop
games over written mail, in a format called ``Play by Post''.

Sessions form the building blocks of the game story or ``Campaign''-
some campaigns may be short enough to be told in a single session (known
as a ``one shot''), whilst others may stretch across multiple sessions
and last months, even years.

Some sessions may involve thrilling battles with rivals and legendary
foes, whilst others may see the party sneaking around town at night, or
bargaining with the local leader for information. What happens during a
sessions is ultimately up to your group to decide- and rarely are two
sessions ever the same.

\newpage

\section{Tools of Play}\label{tools-of-play}

In addition to the rules in this book and some friends, you will need
some other materials to play this game. You can easily find these by
shopping online, or at a local game store.

\begin{itemize}
\item
  \textbf{The Character Sheet:} The character sheet is where each player
  will note down information pertaining to their character, and is
  continuously updated during play. A template character sheet is
  included with this book here (TODO).
\item
  \textbf{Dice:} The players and the GM will each need a twenty-sided
  die (a ``d20''), as well as some six-sided dice (a ``d6''). Having 5
  or more d6s is recommended.

  \begin{itemize}
  \tightlist
  \item
    \textbf{A Coin\ldots?} In some instances, the players or GM may be
    asked to flip a coin. If you'd rather avoid doing that, a d6 can be
    used for this purpose- roll the die, and treat even numbers as
    heads, and odd numbers as tails.
  \end{itemize}
\item
  \textbf{Adventures:} Your table will obviously need an adventure to
  play. Your GM can create their own adventure, or alternatively, they
  can adapt an adventure published by someone else online.
\item
  \textbf{Maps and Tokens:} Having a map of the environment, and tokens
  that can represent the player characters, is a good way to keep track
  of dungeon crawls and the state of combat. Maps typically are marked
  with a 1-inch grid, with each square representing 1 meter in game.
  Tokens are used to represent the positions of players and other
  characters.
\end{itemize}

\part{Player Characters}

\chapter{The Role of the Players}\label{the-role-of-the-players}

The primary responsibility of the players is to take control of a Player
Character, making decisions that would be appropriate for your
characters personality, with the overall goal of overcoming the
challenges set by the Game Master in order to achieve your characters
goals.

Before a campaign begins, each player should sit down and create their
character. This involves both narrative elements, such as creating a
history, personality, and goals for their character to accomplish, as
well as rules-driven elements like choosing your character's type,
species, archetype, and more.

Furthermore, it is the responsiblity of the players to respect the
choices of the GM, and the effort that they put into managing the table
and the narrative. The players and the GM are not always adversaries-
rather, they are working together in order to tell a fulfilling story
and have fun. Play in good faith, and strive to co-operate, rather than
compete.

\chapter{Player Characters}\label{player-characters-1}

Player characters are defined by a combination of their narrative
elements, like their backstory, personality, and goals, as well as the
mechanical rules that control the effectiveness of their choices during
a session. These rules elements, known as ``Statistics'' or ``Stats'',
and the impacts they have on Pokemon characters, are outlined in this
section. You will fill out all of these Game Statistics when completing
your character sheet.

\section{Game Statistics}\label{game-statistics}

\subsection{Ability Scores}\label{ability-scores}

Ability scores are a measurement of your characters core strengths and
weaknesses, both physically and mentally. There are six different
ability scores;

\begin{itemize}
\tightlist
\item
  \textbf{Strength:} Abreviated as ``STR''.
\item
  \textbf{Dexterity} Abreviated as ``DEX''.
\item
  \textbf{Constitution:} Abreviated as ``CON''.
\item
  \textbf{Intelligence:} Abreviated as ``INT''.
\item
  \textbf{Wisdom:} Abreviated as ``WIS''.
\item
  \textbf{Charisma:} Abreviated as ``CHA''.
\end{itemize}

Each of a characters ability scores can vary between a value of 3 (the
worst) and 18 (the best).

\subsection{Type}\label{type}

A ``Type'' is a property of both Pokemon and their moves that influence
how effective the later are when used during battle, and grant other
passive benefits known as Talents. Each Move has a single defined Type,
whilst Pokemon can have one type, or two different types.

\subsection{Archetype}\label{archetype}

The role that your Pokemon plays in their team. A Pokemon's archetype
grants additional Talents to support them during their adventures- with
further power being granted as your Pokemon's Rank increases.

\subsection{Rank}\label{rank}

An indication of a Pokemon's experience as an adventurer. Pokemon grow
stronger as their Rank increases, with Rank 1 Pokemon being fresh
greenhorns, and Rank 10 Pokemon being capable of nigh-legendary feats.

A table of the ranks and the benefits bestowed to a Pokemon if they meet
them is shown below;

\begin{longtable}[]{@{}lllll@{}}
\toprule\noalign{}
Rank & Title & Move Tier & Class Talents & Evolution Level \\
\midrule\noalign{}
\endhead
\bottomrule\noalign{}
\endlastfoot
1 & Rookie & 1 & 1 & Basic \\
2 & Normal & 1 & 2 & Basic \\
3 & Bronze & 2 & 2 & Basic \\
4 & Silver & 2 & 3 & Stage 1 \\
5 & Gold & 3 & 3 & Stage 1 \\
6 & Diamond & 3 & 4 & Stage 1 \\
7 & Super & 4 & 4 & Stage 2 \\
8 & Ultra & 4 & 5 & Stage 2 \\
9 & Hyper & 5 & 5 & Stage 2 \\
10 & Master & 5 & 6 & Stage 2 \\
\end{longtable}

\begin{tcolorbox}[enhanced jigsaw, toptitle=1mm, leftrule=.75mm, bottomrule=.15mm, colbacktitle=quarto-callout-note-color!10!white, opacitybacktitle=0.6, toprule=.15mm, colframe=quarto-callout-note-color-frame, breakable, rightrule=.15mm, titlerule=0mm, left=2mm, coltitle=black, title=\textcolor{quarto-callout-note-color}{\faInfo}\hspace{0.5em}{Note}, bottomtitle=1mm, opacityback=0, arc=.35mm, colback=white]

Whilst a list of titles corresponding to different ranks is provided,
GM's a free to use their own Rank titles if they come up with titles
that better fit the story of their campaign. And of course, per rule
number one, if the table doesn't want to bother remembering specific
titles, they're free to refer to ranks by their associated number
instead.

\end{tcolorbox}

\subsection{Experience Points}\label{experience-points}

Also known as ``EXP''. An indication of the progress a Pokemon has made
on being promoted to the next rank. EXP is awarded by the GM after the
players successfully complete an adventure. When enough EXP has gained,
the Player Characters rank up, and gain new Moves and Talents if
applicable.

\subsection{Talents}\label{talents}

Additional powers that a Pokemon can call upon during their adventure.
Some talents are granted to a Pokemon based on their typing, whilst
others are granted by a Pokemon's archetype.

\subsection{Moves}\label{moves}

Natural yet nonetheless fantastic powers that Pokemon can call upon to
accomplish various tasks. Pokemon can know up to four moves at a given
time. The pool of moves a Pokemon can learn is determined based on their
type and current rank.

\chapter{Character Creation}\label{character-creation}

\chapter{Character Progression}\label{character-progression}

\part{Playing the Game}

\chapter{}\label{section}

\chapter{}\label{section-1}

\chapter{}\label{section-2}

\chapter{}\label{section-3}

\chapter{}\label{section-4}

\chapter{}\label{section-5}

\chapter{}\label{section-6}

\chapter{}\label{section-7}

\part{Moves, Effects, and Conditions}

\chapter{}\label{section-8}

\chapter{}\label{section-9}

\chapter{}\label{section-10}

\chapter{}\label{section-11}

\chapter{}\label{section-12}

\cleardoublepage
\phantomsection
\addcontentsline{toc}{part}{Appendices}
\appendix

\chapter*{Test Page}\label{sec-intro}
\addcontentsline{toc}{chapter}{Test Page}

\markboth{Test Page}{Test Page}

\begin{Shaded}
\begin{Highlighting}[]
\DecValTok{1} \SpecialCharTok{+} \DecValTok{1}
\end{Highlighting}
\end{Shaded}

\begin{verbatim}
[1] 2
\end{verbatim}

\begin{tcolorbox}[enhanced jigsaw, toptitle=1mm, leftrule=.75mm, bottomrule=.15mm, colbacktitle=quarto-callout-note-color!10!white, opacitybacktitle=0.6, toprule=.15mm, colframe=quarto-callout-note-color-frame, breakable, rightrule=.15mm, titlerule=0mm, left=2mm, coltitle=black, title=\textcolor{quarto-callout-note-color}{\faInfo}\hspace{0.5em}{Note}, bottomtitle=1mm, opacityback=0, arc=.35mm, colback=white]

note

\end{tcolorbox}

\begin{tcolorbox}[enhanced jigsaw, toptitle=1mm, leftrule=.75mm, bottomrule=.15mm, colbacktitle=quarto-callout-warning-color!10!white, opacitybacktitle=0.6, toprule=.15mm, colframe=quarto-callout-warning-color-frame, breakable, rightrule=.15mm, titlerule=0mm, left=2mm, coltitle=black, title=\textcolor{quarto-callout-warning-color}{\faExclamationTriangle}\hspace{0.5em}{Warning}, bottomtitle=1mm, opacityback=0, arc=.35mm, colback=white]

warning

\end{tcolorbox}

\begin{tcolorbox}[enhanced jigsaw, toptitle=1mm, leftrule=.75mm, bottomrule=.15mm, colbacktitle=quarto-callout-important-color!10!white, opacitybacktitle=0.6, toprule=.15mm, colframe=quarto-callout-important-color-frame, breakable, rightrule=.15mm, titlerule=0mm, left=2mm, coltitle=black, title=\textcolor{quarto-callout-important-color}{\faExclamation}\hspace{0.5em}{Important}, bottomtitle=1mm, opacityback=0, arc=.35mm, colback=white]

This is important!

\end{tcolorbox}

\begin{tcolorbox}[enhanced jigsaw, toptitle=1mm, leftrule=.75mm, bottomrule=.15mm, colbacktitle=quarto-callout-tip-color!10!white, opacitybacktitle=0.6, toprule=.15mm, colframe=quarto-callout-tip-color-frame, breakable, rightrule=.15mm, titlerule=0mm, left=2mm, coltitle=black, title=\textcolor{quarto-callout-tip-color}{\faLightbulb}\hspace{0.5em}{This note has a title!}, bottomtitle=1mm, opacityback=0, arc=.35mm, colback=white]

It's also a tip!

\end{tcolorbox}

\begin{tcolorbox}[enhanced jigsaw, toptitle=1mm, leftrule=.75mm, bottomrule=.15mm, colbacktitle=quarto-callout-caution-color!10!white, opacitybacktitle=0.6, toprule=.15mm, colframe=quarto-callout-caution-color-frame, breakable, rightrule=.15mm, titlerule=0mm, left=2mm, coltitle=black, title=\textcolor{quarto-callout-caution-color}{\faFire}\hspace{0.5em}{Expand To Learn About Collapse}, bottomtitle=1mm, opacityback=0, arc=.35mm, colback=white]

This is an example of a `folded' caution callout that can be expanded by
the user. You can use \texttt{collapse="true"} to collapse it by default
or \texttt{collapse="false"} to make a collapsible callout that is
expanded by default.

\end{tcolorbox}

\chapter{Glossary}\label{glossary}


\backmatter


\end{document}
